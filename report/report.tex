\documentclass{article}
\usepackage[french]{babel}
\usepackage[utf8]{inputenc}
\usepackage{graphicx}
\usepackage{float}
\usepackage{caption}
\usepackage{subcaption}
\usepackage{listings}
\usepackage{color}
\usepackage{hyperref}
\usepackage[table]{xcolor}
\usepackage{listings}
\usepackage{amsmath}
\usepackage{algorithm}
\usepackage{algorithmic}

\usepackage{fancyhdr} % en-tête
\setlength{\headheight}{14.5pt}
\addtolength{\topmargin}{-2.5pt}



\makeatletter
\renewcommand\thefigure{\thesection.\arabic{figure}}
\@addtoreset{figure}{section}
\makeatother

\begin{document}
\begin{titlepage}

\begin{figure}[h!]
\centering
\includegraphics[scale=1]{download.png}
\end{figure}

\begin{center}
    \fontsize{32}{32}\selectfont\textbf{Compte rendu projet Robot programming}\\[3cm] 
    {\fontsize{15}{15}\selectfont \textbf{AHBAIZ Mouhcine ,NEKT Hicham\\ BOZ Atman, KABOURI Mohamed Yassine.}}     \\   [6cm]
\end{center} 

\end{titlepage}

\newpage
\tableofcontents
\newpage


Ne décrivez pas les algos mais parler du problème.

\section{Présentation du sujet}

C'est quoi robozulle ? le but du projet ?


\section{Définition du cadre de travail}
\subsection{Support}

\paragraph{}Le projet est réalisé entièrement en langage JavaScript et Git comme logiciel de gestion de versions. 
Il met en œuvre les connaissances que l’on a acquit dans les cours d’algorithmiques, programmation fonctionnelle et 
atelier de programmation. Nous avons utilisé JSON et eslint pour mettre en vert la forge. 
%%Enfin, ce projet a imposé un travaille en équipe ce qui a permis à chacun d'apprendre de l'autre. 

\subsection{Responsabilité de chaque membre du groupe}

\section{Algorithme d'itération}
\paragraph{}

Step and exécution

parler des conditions limites aussi

\section{Afficheur pour l'exécution}
\subsection{Afficheur texte}

\subsection{Afficheur html}

\section{Description des tests de validation}

\section{Évaluation des problèmes de mise en œuvre et
de la réalisation}
-pureté des fonctions
\subsection{Première implémentation sans la structure pile}

\subsection{Complexité de nos algorithmes}

\section{Conclusion}
\end{document}
